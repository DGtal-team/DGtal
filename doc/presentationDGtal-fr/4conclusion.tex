\section{Conclusion et Milestones}

\begin{frame}
\frametitle{Conclusion}



\begin{block}{Points positifs\HH}
  
  \begin{itemize}
  \item Multi-site
  \item Design g�n�rique, extensible, multiplateforme
  \item Infrastructure et algorithmes de base performants
  \item M�canisme de visualisation vectorielle 2D  pratique
  \end{itemize}
\end{block}


\begin{exampleblock}{Perspective\HH}
  
  \begin{itemize}
  \item Int�grer de nouveaux sites/d�veloppeurs
  \item Int�grer de nouveaux algorithmes
  \item Construire une documentation orient�e utilisateurs
  \item Poursuivre la construction de ``shortcuts''
  \item Cr�er des d�monstrateurs 
  \item \ldots
  \end{itemize}
\end{exampleblock}

\end{frame}

\begin{frame}
\frametitle{Versions futures}

0.3
\begin{itemize}
\item Analyse volum�trique en distance compl�te (RDT, MA)
\item Espace de Khalimsky (mod�le inter-pixel) et operateurs de bord
\item Couverture tangentielle et estimateurs g�om�triques 2D
\end{itemize}

0.4
\begin{itemize}
\item Conteneurs d'image avanc�s (tuil�s, backends VTK-ITK,...)
\item Mod�le topologique de partitions (ex. carte combinatoire)
\item Plus de primitives g�om�triques
\item Analyse surfacique en dimension 3 et plus
\end{itemize}

mais tout d�pend de vous...

\end{frame}



\begin{frame}
  \Large
  \begin{center}
    \includegraphics[width=7cm]{beamerthemelirislogosmall.png} \\~ ~  \\	
    \url{http://liris.cnrs.fr/dgtal}\\~ ~  \\
    \url{dgtal-users@lists.gforge.liris.cnrs.fr}\\
    \url{dgtal-devel@lists.gforge.liris.cnrs.fr}
    
  \end{center}
\end{frame}
