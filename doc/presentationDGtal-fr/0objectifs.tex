
\section{Objectifs}

\begin{frame}
\frametitle{Objectifs}

  \begin{block}{Biblioth�que G�om�trie discr�te\HH}
    \begin{itemize}
    \item faciliter l'appropriation de nos outils pour un n�ophyte
      (nouveau doctorant, chercheur d'une autre discipline, ...)
    \item tester rapidement une nouvelle id�e, permettre une meilleure
      comparaison d'un nouvel outil par rapport � l'existant
    \item faciliter la construction de d�monstrateurs (statiques, en
      ligne, ...)
    \item diffuser nos r�sultats de recherche � d'autres domaines
      \item mettre en place un projet f�d�rateur
\item \ldots
    \end{itemize}
  \end{block}
 

  \begin{block}{Qui ?\HH}
    \begin{itemize}
    \item LIRIS      
    \item LAMA (Chamb�ry)
    \item Gipsa-lab (Grenoble)
    \item LORIA (Nancy)
    \item GREYC (Caen)
    \end{itemize}
  \end{block}
\end{frame}

\begin{frame}
  \frametitle{Objectifs (bis)}

  
  \begin{block}{Pourquoi faire ?\HH}
    \begin{itemize}
    \item D�finir des objets discrets en dimension arbitraire
    \item Proposer des algorithmes d'analyse geom. et topo
    \item D�finir des m�canismes d'I/O et de visualisation
    \end{itemize}
  \end{block}
\end{frame}

